\documentclass[journal, a4paper]{IEEEtran}

% some very useful LaTeX packages include:
\usepackage[brazil]{babel}
\usepackage[utf8x]{inputenc}
\usepackage{amsmath}
\usepackage{float}
\usepackage{mathtools}
\usepackage{natbib}
%\usepackage{multicol, blindtext}

%\usepackage{cite}      % Written by Donald Arseneau
                        % V1.6 and later of IEEEtran pre-defines the format
                        % of the cite.sty package \cite{} output to follow
                        % that of IEEE. Loading the cite package will
                        % result in citation numbers being automatically
                        % sorted and properly "ranged". i.e.,
                        % [1], [9], [2], [7], [5], [6]
                        % (without using cite.sty)
                        % will become:
                        % [1], [2], [5]--[7], [9] (using cite.sty)
                        % cite.sty's \cite will automatically add leading
                        % space, if needed. Use cite.sty's noadjust option
                        % (cite.sty V3.8 and later) if you want to turn this
                        % off. cite.sty is already installed on most LaTeX
                        % systems. The latest version can be obtained at:
                        % http://www.ctan.org/tex-archive/macros/latex/contrib/supported/cite/

\usepackage{graphicx}   % Written by David Carlisle and Sebastian Rahtz
                        % Required if you want graphics, photos, etc.
                        % graphicx.sty is already installed on most LaTeX
                        % systems. The latest version and documentation can
                        % be obtained at:
                        % http://www.ctan.org/tex-archive/macros/latex/required/graphics/
                        % Another good source of documentation is "Using
                        % Imported Graphics in LaTeX2e" by Keith Reckdahl
                        % which can be found as esplatex.ps and epslatex.pdf
                        % at: http://www.ctan.org/tex-archive/info/

\usepackage{psfrag}    % Written by Craig Barratt, Michael C. Grant,
                        % and David Carlisle
                        % This package allows you to substitute LaTeX
                        % commands for text in imported EPS graphic files.
                        % In this way, LaTeX symbols can be placed into
                        % graphics that have been generated by other
                        % applications. You must use latex->dvips->ps2pdf
                        % workflow (not direct pdf output from pdflatex) if
                        % you wish to use this capability because it works
                        % via some PostScript tricks. Alternatively, the
                        % graphics could be processed as separate files via
                        % psfrag and dvips, then converted to PDF for
                        % inclusion in the main file which uses pdflatex.
                        % Docs are in "The PSfrag System" by Michael C. Grant
                        % and David Carlisle. There is also some information
                        % about using psfrag in "Using Imported Graphics in
                        % LaTeX2e" by Keith Reckdahl which documents the
                        % graphicx package (see above). The psfrag package
                        % and documentation can be obtained at:
                        % http://www.ctan.org/tex-archive/macros/latex/contrib/supported/psfrag/

\usepackage{subfig} % Written by Steven Douglas Cochran
                        % This package makes it easy to put subfigures
                        % in your figures. i.e., "figure 1a and 1b"
                        % Docs are in "Using Imported Graphics in LaTeX2e"
                        % by Keith Reckdahl which also documents the graphicx
                        % package (see above). subfigure.sty is already
                        % installed on most LaTeX systems. The latest version
                        % and documentation can be obtained at:
                        % http://www.ctan.org/tex-archive/macros/latex/contrib/supported/subfigure/

\usepackage{url}        % Written by Donald Arseneau
                        % Provides better support for handling and breaking
                        % URLs. url.sty is already installed on most LaTeX
                        % systems. The latest version can be obtained at:
                        % http://www.ctan.org/tex-archive/macros/latex/contrib/other/misc/
                        % Read the url.sty source comments for usage information.

\usepackage{stfloats}  % Written by Sigitas Tolusis
                        % Gives LaTeX2e the ability to do double column
                        % floats at the bottom of the page as well as the top.
                        % (e.g., "\begin{figure*}[!b]" is not normally
                        % possible in LaTeX2e). This is an invasive package
                        % which rewrites many portions of the LaTeX2e output
                        % routines. It may not work with other packages that
                        % modify the LaTeX2e output routine and/or with other
                        % versions of LaTeX. The latest version and
                        % documentation can be obtained at:
                        % http://www.ctan.org/tex-archive/macros/latex/contrib/supported/sttools/
                        % Documentation is contained in the stfloats.sty
                        % comments as well as in the presfull.pdf file.
                        % Do not use the stfloats baselinefloat ability as
                        % IEEE does not allow \baselineskip to stretch.
                        % Authors submitting work to the IEEE should note
                        % that IEEE rarely uses double column equations and
                        % that authors should try to avoid such use.
                        % Do not be tempted to use the cuted.sty or
                        % midfloat.sty package (by the same author) as IEEE
                        % does not format its papers in such ways.

\usepackage{amsmath}    % From the American Mathematical Society
                        % A popular package that provides many helpful commands
                        % for dealing with mathematics. Note that the AMSmath
                        % package sets \interdisplaylinepenalty to 10000 thus
                        % preventing page breaks from occurring within multiline
                        % equations. Use:
\interdisplaylinepenalty=2500
                        % after loading amsmath to restore such page breaks
                        % as IEEEtran.cls normally does. amsmath.sty is already
                        % installed on most LaTeX systems. The latest version
                        % and documentation can be obtained at:
                        % http://www.ctan.org/tex-archive/macros/latex/required/amslatex/math/



% Other popular packages for formatting tables and equations include:

%\usepackage{array}
% Frank Mittelbach's and David Carlisle's array.sty which improves the
% LaTeX2e array and tabular environments to provide better appearances and
% additional user controls. array.sty is already installed on most systems.
% The latest version and documentation can be obtained at:
% http://www.ctan.org/tex-archive/macros/latex/required/tools/

% V1.6 of IEEEtran contains the IEEEeqnarray family of commands that can
% be used to generate multiline equations as well as matrices, tables, etc.

% Also of notable interest:
% Scott Pakin's eqparbox package for creating (automatically sized) equal
% width boxes. Available:
% http://www.ctan.org/tex-archive/macros/latex/contrib/supported/eqparbox/

% *** Do not adjust lengths that control margins, column widths, etc. ***
% *** Do not use packages that alter fonts (such as pslatex).         ***
% There should be no need to do such things with IEEEtran.cls V1.6 and later.


% Your document starts here!
\begin{document}

% Define document title and author
	\title{Relatório sobre o Mapa de Kohonen}
	\author{Victor Carreira
	\thanks{Professora: Marley. Eng. Elétrica. PUC-RIO}}
	\markboth{Trabalho 03}{}
	\maketitle

% Write abstract here
\begin{abstract}

\end{abstract}


\section{Introdução}
    As Redes Neurais Artificiais (RNA) são inspiradas em modelos sensoriais do processamento de tarefas realizadas pelo cérebro \citep{Hagan1996}. Uma RNA, portanto pode ser criada através da aplicação de algoritmos matemáticos que imitem a tarefa realizada por um neurônio \citep{Nedjah2016}. Uma rede neural artificial possui semelhanças com a rede biológica presente no sistema nervoso central, neste o cômputo de informações realizado do cérebro é feito através de uma vasta quantidade de neurônios interconectados \citep{Feldman1988,Poulton2002}. A comunicação entre essas células é realizada através de impulsos elétricos. Estes são transmitidos e recebidos por meio de sinapses nervosas entre axônios e dendritos. As sinapses são estruturas elementares e uma unidade funcional localizada entre dois neurônios \citep{Krogh2008}.

	\citet{McCulloch1943} redigem o trabalho pioneiro onde foi modelado um neurônio cuja resposta dependia do \textit{input}\footnote{Valor de entrada} que provinha de outros neurônios e do peso utilizado.  Já \citet{Rosenblatt1962} cria a teoria de convergência do \textit{Perceptron} onde ele prova que modelos de neurônios possuem propriedades similares ao cérebro humano \citep{Kanal2001}. Neste sentido as rede neuronais artificiais podem realizar performasses sofisticadas no reconhecimento de padrões, mesmo se alguns neurônios forem destruídos \citep{Levy1997}. \citet{Minsky1969} demonstraram que um único  \textit{Perceptron} somente resolve uma classe muito limitada de problemas que podem ser linearizados.
	

	
	Neste relatório são apresentados os resultados do Trabalho 03 mapas de Kohonen da disciplina ELE 2394, Redes Neurais I, da Engenharia elétrica.
	

% Main Part
\section{Objetivo e Metodologia}


Em um hospital na Austrália, 215 pacientes foram submetidos a 5 testes de laboratórios. Testes adicionais, como por exemplo o exame clínico, permitiram determinar se os pacientes tinham eutiroidismo (normal), hypotiroidismo ou hypertiroidismo. Os resultados dos 5 testes e a condição da tiróide de cada paciente estão na base de dados new-thyroid.dat cujas colunas representam:

\begin{enumerate}
\item Percentual da resina-T3.
\item Tiroxina total.
\item Triiodotironina total.
\item Hormônio estimulador da tiróide (TSH)
\item  Diferença absoluta máxima no valor da TSH após a injeção de 200 micro gramas de hormônio de liberação de tirotropina.  
\item Classe (1 = normal, 2 = hyper, 3 = hypo).
\end{enumerate}


O objetivo é, utilizando mapas de Kohonen, agrupar os diferentes tipos de pacientes e em seguida determinar o perfil de cada grupo obtido. Sendo assim, o trabalho está divido em duas partes: configuração do mapa e análise dos resultados.

\begin{itemize}[Configuração do mapa:]
\item[a] Topologia
\item[b] Tipo de Normalização dos dados de entrada
\item[c] Tempo de treinamento durante a fase de ordenação
\item[d] Tempo de treinamento durante a fase de ajuste fino
\end{itemize}

Para os parâmetros que não listados, deve-se escolher um valor e explicar a escolha. Analise os resultados obtidos e determine a melhor configuração obtida.

\begin{itemize}[Análise dos resultados:]
\item[e] Utilizando a melhor configuração do item 1, analise os diferentes mapas fornecidos.
\item[f] Caracterize os grupos de pacientes obtidos.
\end{itemize}


\section{Princípio Teórico}

Os primeiros mapas auto-organizáveis foram propostos por \citet{Malsburg1976}.

%O neurônio de \citet{McCulloch1943} propõe um limite binário para a criação de um modelo. Este neurônio artificial registra uma soma de pesos de $n$ sinais de entrada, $x_{j}$, $j=1,2,3,...,n$, e fornece um \textit{output}\footnote{Valor de saída} de $1$ caso esta soma esteja acima do limite $u$. Caso contrário o \textit{output} é $0$. Matematicamente essa relação pode ser descrita de acordo com a Eq. \ref{Eq.neuronio-McCulloch}:
%
%\begin{eqnarray}
%y=\theta \left( \sum^{n}_{j=1} w_{j} x_{j} -u \right)
%\label{Eq.neuronio-McCulloch}
%\end{eqnarray}
%
%Onde $\theta$ é o passo dado na posição $0$, $w_{j}$ é chamada sinapse-peso associado a um $j_{esimo}$ \textit{input}. A título de simplificação a função limite\footnote{Genericamente chamada de função de ativação} $u$ é considerada um outro peso $w_{0}=-u$ anexado a um neurônio com um \textit{input} constante $x_{0}=1$. Pesos positivos correspondem a uma sinapse \textbf{excitatória}, enquanto pesos negativos correspondem a uma sinapse \textbf{inibitória}. Este modelo contém uma série de simplificações que não refletem o verdadeiro comportamento dos neurônios biológicos \citep{Mao1996}.  
%
%Derivações do neurônio de \citet{McCulloch1943} na escolha das funções de ativação. Uma função largamente utilizada é a função sigmóide, que exibe uma suavização dos \textit{outputs} a medida que o valor da função diminui \citep{Mao1996,Misra2010}. Essa função de ativação pode ser expressa de acordo com a Eq. \ref{f.sigmoide}:
%
%\begin{eqnarray}
%g(x)=1/(1+e^{-\beta x})
%\label{f.sigmoide}
%\end{eqnarray}
%
%Onde $\beta$ é o parâmetro de inclinação. %A Fig. \ref{Esquematico de McCulloch} ilustra a sequência lógica da operação de uma RNA para um neurônio simples de McCulloch-Pitts. 
%\\
%\begin{figure*}[!ht]
%	\centering
%	\setlength{\fboxsep}{8pt}
%	\setlength{\fboxrule}{0.1pt}
%	\fbox{
%		\includegraphics[scale=0.8]{Images/McCulloch.eps}
%	}
%	\caption{Modelo esquemático de um neurônio de McCulloch-Pitts. Onde $x_{1}, x_{2}, ..., x_{n}$ são os \textit{inputs}, $w_{1}, w_{2}, ..., w_{n}$ são os pesos, h é o treino, $g(x)$ é a função de ativação, e $y$ é o \textit{output}.}
%	\label{Esquematico de McCulloch}
%\end{figure*}




\section{Resultados}

\begin{figure*}
	\centering
	\includegraphics[scale=0.5]{Images/SOM1.png}
	\label{SOM1}
	\caption{Primeiro teste.}
\end{figure*}

\begin{figure*}
	\centering
	\includegraphics[scale=0.5]{Images/SOM2.png}
	\label{SOM2}
	\caption{Segundo teste.}
\end{figure*}

\begin{figure*}
	\centering
	\includegraphics[scale=0.5]{Images/SOM3.png}
	\label{SOM3}
	\caption{Terceiro teste.}
\end{figure*}

\begin{figure*}
	\centering
	\includegraphics[scale=0.5]{Images/SOM4.png}
	\label{SOM4}
	\caption{Quarto teste.}
\end{figure*}

\section{Conclusões}




% Now we need a bibliography:


\bibliographystyle{apalike}
\bibliography{references}


% Your document ends here!
\end{document}